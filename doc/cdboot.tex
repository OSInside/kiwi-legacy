\index{KIWI images!cdboot|(}
\chapter{Creating a LiveCD from an image}
\label{chapter:cdboot}
\minitoc

\section{What is a live CD}
Normally an image will be installed on a disk or into the
main memory of a computer. This is done by a deployment
architecture which transfers the image via a boot image
into its final destination. such a boot image can also
exist on a CD. The task of the CD-Boot image is to setup
a system which is divided into two parts:

\begin{itemize}
	\item Read-Only part which is obtained from the CD
	\item Read/Write part which is pushed into main memory
\end{itemize}

An image which works partially on disk and partially on RAM
is called a live CD system. The CD-Boot structure of KIWI will
put the directories \textit{/bin, /boot, /lib, /opt, /sbin and /usr}
on the CD and the rest into the main memory of the system.

\subsection{How do you setup an image as live CD}
To create an .iso image which can be burned on CD you only need
to specify the boot image which should handle your live system.
This is done by setting the \textbf{type} of the image in the
\textbf{config.xml} file as follows:

\begin{Command}{9cm}
<preferences>\\
\hspace*{1cm}<type>iso:name</type>\\
\hspace*{1cm}...\\
</preferences>
\end{Command}

The parameter \textbf{name} refers to the CD boot image which
must exist in \textit{/usr/share/kiwi/image}. Like for all boot
images the most important point is that the boot image has to
match the operating system image. This means the kernel of the boot-
and operating system image must be the same. If there is no
boot image which matches you need to create your own boot image
description. A good starting point for this is to use an existing
boot image and adapt it to your needs.

\section{How does the boot image work}
The boot process from a CD works in five steps:

\begin{enumerate}
\item After the image has been prepared which means a physical extend
      exists the process of building the logical extend will split
      the physical part into two physical extends:
      \begin{itemize}
      \item read/write extend which is loaded into RAM
      \item read-only extend consisting of the data stored
            in the \textit{/bin, /boot, /lib, /opt, /sbin and /usr}
            directories
      \end{itemize}
\item From the two physical extends kiwi will create two ext2
      based logical images.
\item Next kiwi prepares and creates in one step the boot image given 
      as the type of the image configuration file config.xml.
\item Next kiwi will setup the CD directory structure and copy all
      image and kernel files as well as the isolinux data into this
      directory tree.
\item At last kiwi calls the \textbf{isolinux.sh} script provided
      by the ISO boot image to create an .iso file from the CD
      directory tree.
\end{enumerate}
