\chapter{Installation Sources and Products}
\label{chapter:instsrc}
\minitoc

A \textbf{Product} or \textbf{Installation Source} is what you find on an
installation disk or in an online repository. In contrast to a live disk the
media contain plain RPM archives and YaST metadata for installation. Furthermore
there is an installation system and kernel which will be booted from the medium
in case it is a disk. For remote installation sources (e.g. a FTP tree) the user
can download the \textit{initrd} and \textit{kernel} and add a boot menu entry
manually to boot the installation system.

TODO Add link/reference how this is done.


\section{Prerequisites}

First of all the installation source creation requires the optional packages
\textit{kiwi-instsource} plus supplementing plugins from the packages
\textit{kiwi-instsource-plugins-openSUSE-<VERSION>-<RELEASE>}
which are all configured to \textit{supplement} the kiwi-instsource
package. This means that multiple repositories might be necessary if the plugin
packages reside elsewhere than the other kiwi packages. The plugins are packaged
in extra packages for some reasons. First of all, the YaST metadata is no fixed
standard but tends to change from release to release, so the code creating the
metadata is version specific. Finally the concept allows users to make their own
custom plugins for special purposes. This mechanism will be explained in more
detail in ref:TODO


\section{Interface}

The interface for installation source creation is implemented in two steps. The
first one is an extension of the existing config syntax as described in
\vref{subsec:config-extend}, the second one is the implementation of easily
configurable plugins for the metadata creation as described in
\vref{subsec:plugins}.


\subsection{Configuration File Extension}
\label{subsec:config-extend}

The config.xml file now supports a new optional element called
\textbf{instsource}. Its purpose is to specify the things that are significant
for an installation repository:
\begin{itemize}
  \item{metadata}
  \item{media structure}
  \item{additional iso information for disc media}
  \item{package signing keys}
\end{itemize}

Furthermore the installation source needs some information for architectures. It
is easily possible to create an installation DVD that boots on 32\,bit and
64\,bit Intel and also PowerPC architecture. In such a case several conditions
must be met:
\begin{itemize}
  \item{the packages must be available for each architecture}
  \item{the disc must boot on each architecture}
  \item{YaST metadata must be available for each architecture}
\end{itemize}
Additionally multiple boot catalog entries must be made in the iso filesystem.


\subsection{Metadata Creation Plugins}
\label{subsec:plugins}

Blah.


\section{Calling Options}

TODO


\section{Directory Layout}

TODO


\section{Logfile Analysis}

TODO


%\section{ISO Generation}



