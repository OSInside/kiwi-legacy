\index{boot images|(}
\index{CRs!booting|(}
\chapter{The Boot Process of a Cash Register System}
\label{CRbootprocess}
To understand how to use the operating system images, a short
description of the cash register system is useful. The
following diagram shows the simplified boot process of a
cash register system.

\begin{figure}[h]
\centering
%\includegraphics[width=10cm]{pictures/crboot}
\caption{Simplified Flow Chart of the Boot System}
\end{figure}

The most important point is to understand the cooperation between
the initial boot image \textbf{initrd.gz} and the intrinsic cash register
image. If the system is able to boot via a network, it will load the kernel
and the compressed boot image from the network. The \textit{brain} of the
boot image is the \textbf{linuxrc} script, which does all the
stuff controlled by an image configuration file also obtained from the
network. The major task is to download and activate the cash register
image. The boot image is exchanged for the cash register image to be
activated.

Normally, only work with the cash register images and leave
the boot image as it is delivered from SUSE.


%\section{Short Overview}
The following short overview describes the steps that take place when the cash register
is booted to the image determined by its product ID:

\begin{itemize}
\item Via PXE network boot or boot manager (GRUB), the cash register boots the
      initrd (initrd.gz) that it receives from the branch server. If no PXE
      boot is possible, the cash register tries to boot from the hard
      disk, if accessible.  
\item Running \textbf{linuxrc} starts the process described below.
      For a more detailed information, refer to Chapter \ref{section:bootcr}
      on page \pageref{section:bootcr}.
\end{itemize}

\begin{enumerate}
      \item The required file systems to receive system data are mounted.
      \item The cash register hardware type is detected.
      \item The cash register bios version is detected.
      \item Network support is activated
                        corresponding to the hardware type of the cash register.
      \item The network interface is set up via DHCP.
      \item The TFTP server address is acquired.
      \item The configuration file \textbf{config.$<$MAC Address$>$}
                        is loaded from the server directory
            \textit{/tftpboot/CR} via TFTP or a new cash register
                        is immediately registered.
      \item The PART: line in the configuration file is analyzed.
      \item The SYNC: line in the configuration file is evaluated.
      \item Indicated images are downloaded with multicast TFTP.
      \item Checksums are checked and download is repeated if necessary.
      \item The CONF: line of the configuration file is evaluated.
      \item All the user-land processes based on the boot image are terminated.
      \item The cash register image is mounted.
      \item The configuration files are copied
            into the mounted cash register image.
      \item The system switches to the mounted cash register image.
      \item The boot image is unmounted.
      \item The kernel initiates
            the \textbf{init} process that starts processing the boot
            scripts.
\end{enumerate}

\index{boot images|)}
\index{CRs!booting|)}
